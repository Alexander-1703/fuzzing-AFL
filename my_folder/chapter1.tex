\chapter{Фаззинг, как методика обнаружения уязвимостей в ПО} \label{ch1}

% не рекомендуется использовать отдельную section <<введение>> после лета 2020 года
%\section{Введение. Сложносоставное название первого параграфа первой главы для~демонстрации переноса слов в содержании} \label{ch1:intro}

Фаззинг-тестирование - это тип тестирования безопасности, который позволяет обнаружить ошибки кодирования и лазейки в программном обеспечении, операционных системах или сетях. Фаззинг включает в себя ввод огромного количества случайных данных в тестируемое программное обеспечение, чтобы заставить его дать сбой. Одним из ключевых преимуществ фаззинга является его способность эффективно сканировать программное обеспечение на предмет ошибок обработки данных, таких как переполнение буфера, уязвимости для SQL-инъекций, кросс-сайтового скриптинга и других видов уязвимостей, которые могут быть эксплуатированы злоумышленниками. 
\par
Фаззинг выполняется с помощью фаззера - программы, которая автоматически вводит псевдослучайные данные в программу и обнаруживает ошибки. Фаззинг-тестирование обычно выполняется автоматически. Оно обнаруживает наиболее серьезные ошибки или дефекты безопасности и является экономически эффективным методом тестирования. Фаззинг также является одним из самых распространенных методов, используемых хакерами для обнаружения уязвимостей системы [1].


\section{Классификация фаззеров} \label{ch1:sec1}


%\subsection{Название первого подпараграфа первого параграфа первой главы для~демонстрации переноса слов в содержании} % ~ нужен, чтобы избавиться от висячего предлога (союза) в конце строки

По методу генерации новых входных значений фаззеры подразделяются на:
\begin{itemize}
	\item Mutation-Based Fuzzers: Это один из типов фаззинга, при котором фаззер имеет некоторые знания о входном формате тестируемой программы: на основе существующих выборок данных инструменты фаззинга на основе мутаций генерируют новые варианты (мутанты), которые он использует для фаззинга. Прим.: AFL++.
	\item Generation-Based Fuzzers: Фаззер генерирует входные данные с нуля. Например, использует входную модель, предоставленную пользователем, для генерации новых входных данных. Прим.: Peach Fuzzer [1].
\end{itemize}

По цели фаззинга:
\begin{itemize}
	\item Protocol-based Fuzzers. Фокусируются на тестировании безопасности сетевых протоколов и программ, обрабатывающих конкретные файловые форматы (изображения, архивы и документы и другое) Прим.: Boofuzz.
	\item Library and Framework Fuzzers. Предназначаются для тестирования функций из готовых библиотек и пакетов. Прим.: LibFuzzer, Jazzer.
	\item Binary Fuzzers. Осуществляют проверку безопасности исполняемого бинарного кода. Прим.: AFL++ (QEMU mode).
	\item API Fuzzers. Предназначаются для тестирования безопасности веб-сервисов и API. Прим.: Restler [6]. 
\end{itemize}



%Одиночные формулы оформляют в окружении \texttt{equation}, например, как указано в следующей одиночной нумерованной формуле:
%
%
%\begin{equation}% лучше не оставлять пропущенную строку (\par) перед окружениями для избежания лишних отсупов в pdf
%\label{eq:Pi-ch1} % eq - equations, далее название, ch поставлено для избежания дублирования
%\pi \approx 3,141.
%\end{equation}
%%
%%
%\begin{figure}[ht!] 
%	\center
%	\includegraphics [scale=0.27] {my_folder/images//spbpu_hydrotower}
%	\caption{Вид на гидробашню СПбПУ \cite{spbpu-gallery}} 
%	\label{fig:spbpu_hydrotower}  
%\end{figure}
%
%
%\begin{table} [htbp]% Пример оформления таблицы
%	\centering\small
%	\caption{Представление данных для сквозного примера по ВКР \cite{Peskov2004}}%
%	\label{tab:ToyCompare}		
%		\begin{tabular}{|l|l|l|l|l|l|}
%			\hline
%			$G$&$m_1$&$m_2$&$m_3$&$m_4$&$K$\\
%			\hline
%			$g_1$&0&1&1&0&1\\ \hline
%			$g_2$&1&2&0&1&1\\ \hline
%			$g_3$&0&1&0&1&1\\ \hline
%			$g_4$&1&2&1&0&2\\ \hline
%			$g_5$&1&1&0&1&2\\ \hline
%			$g_6$&1&1&1&2&2\\ \hline		
%		\end{tabular}	
%	\normalsize% возвращаем шрифт к нормальному
%\end{table}


% \firef{} от figure reference
% \taref{} от table reference
% \eqref{} от equation reference

%На \firef{fig:spbpu_hydrotower} изображена гидробашня СПбПУ, а в \taref{tab:ToyCompare} приведены данные, на примере которых коротко и наглядно будет изложена суть ВКР.


\section{Недостатки fuzzing-тестирования} \label{ch1:sec2} 
Несмотря на очевидные преимущества, фаззинг также имеет недостатки:
\begin{itemize}
	\item Само по себе fuzzing тестирование не может дать полную картину общей угрозы безопасности или ошибок.
	\item Fuzzing тестирование менее эффективно для борьбы с угрозами безопасности, которые не вызывают сбоев программы, например, с некоторыми вирусами, червями, троянами и т. д.
	\item Fuzzing тестирование может обнаружить только простые неисправности или угрозы.
	\item Для эффективной работы потребуется значительное время [7].
\end{itemize}
\par
 Тем не менее, фаззинг остается одним из самых мощных инструментов в арсенале специалистов по кибербезопасности для обнаружения и устранения уязвимостей в программном обеспечении. Применение этой методики позволяет не только улучшить качество разрабатываемых программ, но и значительно повысить уровень защиты информационных систем от внешних атак.
 
%Формулы могут быть размещены в несколько строк. Чтобы выставить номер формулы напротив средней строки, используйте окружение \verb|multlined| из пакета \verb|mathtools| следующим образом \cite{Ganter1999}:
%%
%\begin{equation} 
%\label{eq:fConcept-order-ch1}
%\begin{multlined}
%(A_1,B_1)\leq (A_2,B_2)\; \Leftrightarrow \\  \Leftrightarrow\; A_1\subseteq A_2\; \Leftrightarrow \\ \Leftrightarrow\; B_2\subseteq B_1. 
%\end{multlined}
%\end{equation}
%
%
%Используя команду \verb|\labelcref| из пакета \verb|cleveref|, допустимо следующим образом оформлять ссылку на несколько формул:
%(\labelcref{eq:Pi-ch1,eq:fConcept-order-ch1}).
%%
%%
%\input{my_folder/tex/fig-spbpu-whitehall-three-in-one} % пример подключения 3х иллюстрации в одном рисунке
%
%Пример ссылок \cite{Article,Book,Booklet,Conference,Inbook,Incollection,Manual,Mastersthesis,Misc,Phdthesis,Proceedings,Techreport,Unpublished,badiou:briefings}, а также ссылок с указанием страниц, на котором отображены номера страниц  \cite[с.~96]{Naidenova2017} или в виде мультицитаты на несколько источников \cites[с.~96]{Naidenova2017}[с.~46]{Ganter1999}. Часть библиографических записей носит иллюстративный характер и не имеет отношения к реальной литературе. 
%
%
%
%%\FloatBarrier % заставить рисунки и другие подвижные (float) элементы остановиться
%\texttt{}
%\section{Выводы} \label{ch1:conclusion}
%
%Текст выводов по главе \thechapter.
%
%Кроме названия параграфа <<выводы>> можно использовать (единообразно по всем главам) следующие подходы к именованию последних разделов с результатами по главам:
%\begin{itemize}
%	\item <<выводы по главе N>>, где N --- номер соответствующей главы;
%	\item <<резюме>>;
%	\item <<резюме по главе N>>, где N --- номер соответствующей главы.
%\end{itemize}
%
%Параграф с изложением выводов по главе \textit{является обязательным}.

%% Вспомогательные команды - Additional commands
%
%\page % принудительное начало с новой страницы, использовать только в конце раздела
%\clearpage % осуществляется пакетом <<placeins>> в пределах секций
%\newpage\leavevmode\thispagestyle{empty}\newpage % 100 % начало новой страницы