\chapter*{Заключение} \label{ch-conclusion}
\addcontentsline{toc}{chapter}{Заключение}	% в оглавление 

\textbf{ПОЛНОСТЬЮ дописываем (переписываем) с указанием также того, что сделал автор: рассмотрел различные виды фаззинга, изучил работу AFL++, развернул и на примере пофаззил, выявил, что не так все хорошо и далее общие слова (если Ваши...)}


Фаззинг, включая AFL++, является мощным инструментом для обнаружения уязвимостей в программном обеспечении с низкими затратами на внедрение и автоматизацией процесса тестирования. Этот метод позволяет обнаруживать широкий спектр ошибок. Благодаря автоматическому созданию и вводу большого количества случайных или псевдослучайных данных фаззинг позволяет проводить тестирование программного обеспечения быстро и эффективно.

Однако у фаззинга есть и ограничения. Во-первых, он имеет ограниченное покрытие кода, что может привести к пропуску некоторых ошибок, особенно если они находятся за пределами тестового покрытия. Во-вторых, некоторые типы ошибок, такие как логические ошибки, могут быть сложными для обнаружения с помощью фаззинга. Также важно отметить, что для достижения оптимальных результатов фаззинг требует дополнительной настройки, включая выбор правильных стратегий мутации и анализ результатов.

В целом, фаззинг, включая инструменты типа AFL++, является хорошим  дополнением к методам тестирования безопасности программного обеспечения. Он позволяет быстро обнаруживать уязвимости с низкими затратами на внедрение, что делает его привлекательным инструментом для разработчиков и исследователей. Однако для обеспечения полной безопасности программного обеспечения рекомендуется использовать фаззинг в сочетании с другими методами тестирования, такими как статический анализ кода и ручное тестирование.
