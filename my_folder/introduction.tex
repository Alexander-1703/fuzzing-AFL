\chapter*{Введение} % * не проставляет номер
\addcontentsline{toc}{chapter}{Введение} % вносим в содержание

В современном мире информационных технологий повышение безопасности программного обеспечения является одной из ключевых проблем. С увеличением сложности программных продуктов и их широким распространением во всех сферах жизни общества, вопрос выявления и устранения уязвимостей становится особенно актуальным. Классические методы тестирования сталкиваются с рядом проблем, таких как невозможность предсказания всех сценариев использования ПО, сложности в обнаружении трудноуловимых ошибок и т.д. Фаззинг, как метод автоматизированного тестирования программ на наличие ошибок и уязвимостей путем подачи неожиданных или случайно сгенерированных данных на вход, показывает высокую эффективность в решении данной проблемы. Однако, несмотря на свои преимущества, фаззинг сталкивается с рядом сложностей, таких как выбор оптимальных стратегий генерации входных данных и анализ большого объема результатов тестирования.
\par
Актуальность применения фаззинга определяется постоянно растущими требованиями к безопасности программного обеспечения и необходимостью эффективного и своевременного обнаружения потенциальных уязвимостей. Методы фаззинга, и в частности инструменты вроде AFL++, предлагают возможность автоматизации процесса поиска ошибок, что значительно повышает шансы на их обнаружение до того, как они будут эксплуатированы пользователями. Важность фаззинга усиливается в контексте непрерывно возрастающего числа угроз информационной безопасности и роста сложности программных систем, что делает традиционные методы тестирования недостаточно эффективными.
\par
Целью данной работы является исследование и анализ методов фаззинга, с акцентом на использование инструмента AFL++ для выявления уязвимостей в программном обеспечении. В рамках работы планируется детально изучить теоретические основы фаззинга, его внутреннее устройство и алгоритмы, а также провести практическое тестирование с использованием fuzz-goat для демонстрации возможностей и эффективности метода. Особое внимание будет уделено анализу результатов фаззинга, выявлению и классификации обнаруженных уязвимостей, а также оценке возможностей AFL++ как инструмента для улучшения безопасности программного обеспечения.
\par
В рамках достижения поставленной цели работы определены следующие задачи:
\begin{enumerate}
	\item Изучение теоретических аспектов фаззинга как методики обнаружения уязвимостей в программном обеспечении, включая историческое развитие, основные подходы и классификацию фаззеров.
	\item Анализ внутреннего устройства и алгоритмов фаззинга, освещение принципов генерации входных данных и механизмов обнаружения ошибок.
	\item Освещение особенностей и возможностей AFL++, включая сравнение с другими инструментами фаззинга и детальное рассмотрение его функциональности.
	\item Проведение практического фаззинга на примере fuzz-goat с использованием AFL++.
	\item Выявление и анализ обнаруженных уязвимостей в ходе фаззинга fuzz-goat.
	\item Оценка эффективности AFL++ и фаззинга в целом как инструментов повышения безопасности программного обеспечения, формулирование выводов о преимуществах и ограничениях методики.
\end{enumerate}

%% Вспомогательные команды - Additional commands
%\newpage % принудительное начало с новой страницы, использовать только в конце раздела
%\clearpage % осуществляется пакетом <<placeins>> в пределах секций
%\newpage\leavevmode\thispagestyle{empty}\newpage % 100 % начало новой строки